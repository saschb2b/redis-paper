\chapter{Data Model}

Redis is a NoSQL database with an aggregate orientation and is likewise key-based \cite{redis} \cite{redis_book}. This means that keys identify the assigned value, this way a client can extract the value by knowing its key. Further you can put a value for a key or delete a key and likewise its value. Redis keys are binary safe which means any binary sequence can be used as a key. An empty string is also a valid key. But Redis is not a plain key-value store. Because a key is binary safe the value can hold more complex data structures. By complex is meant the possibility to nest values in values or even map objects. 

\section{Data Structures}
Redis supports the following data structures:
\begin{itemize}  
\item Strings: a String-value can have a total length of 512 Mb, assembling Strings
\item Lists: simply an ordered list of Strings => a sequence with duplicates, adding 	new elements on the head (left) or on the tail (right), max length of a list is 232 (around 4 billion of elements per list
\item Hashes: maps between String-fields and String-values, mainly represent objects, have a max length of 232
\item Sets: an ordered collection of Strings, not allowing duplicates (if u add the 	same element multiple times will result in having just a single copy of this element), max number of elements is the same as lists, you can apply commands or set operations
\item Sorted Sets: similar to Set but associated with a score ordering the sorted set from the smallest to the greatest score, elements are unique but scores may be repeated
\item Bitmaps and HyperLogLogs: basing on Strings with their own semantic Message Queues 	
\end{itemize}

You can run atomic operations on each of these data types. With SET you set a specific value for a key and with GET you retrieve this value. SET replaces any existing value already stored into a key in the case this key already exists. This way SET can be used to update a value of an existing key. Because the value of the key is overwritten with a new value, the old value is kind of deleted. To delete a key with its associated value you run the command DEL. Redis offers the possibility to set or retrieve the value of multiple keys in a single command using MSET and MGET. MGET returns an array of values.

Redis supports around 50 programing languages. 12 languages are particular recommended. For Java there are 4 different libraries. A few examples for different programming languages in the following are JRedis for Java , Redis-rb for Ruby, Credis for C and Redis-py for Python.